 
Rasdaman\newline 
Multi-Dimensional Raster Database\newline 

Rasdaman is an implementation of the array database model that extends standard relational databases, such as PostgreSQL, to provide storage and retrieval of multi-dimensional raster data of unlimited size. Data can be stored and retrieved using an SQL-style raster query language, with highly effective server-side optimization. Its petascope component provides also web based interfaces to the data suitable for geospatial applications and based on OGC stadards such as WCS, WCPS, and WPS. Programmer APIs are also available for C++ and Java languages. A rasdaman driver is a part of the GDAL (Geospatial Data Abstraction Library) library for geospatial data formats, a MapServer integration is available in beta.
\newline 
The rasdaman technology is stable and mature, deployed in production since over 10 years; the French National Geographic Institute runs rasdaman on a dozen-Terabyte airborne image map. At the ACM Principles of Database Systems Conference in 2007, raster database expert Rona Machlin characterizes rasdaman as “the most comprehensive implementation of such a system”.
\newline 

Core Features
\newline 
\begin{itemize}
 \item true multi-dimensionality - from 1-D over 2-D to 3-D, 4-D, and beyond
 \item powerful, flexible, SQL-style query language for tasks such as: visualization, classification, convolution, aggregation, and many more geospatial functions
 \item spatial indexing and adaptive tiling for fast data access
 \item tile streaming for scalability and high performance on moderate hardware
 \item multi-user support through server multiplexing
 \item full information integration of raster data with all other geographic data in the PostgreSQL databases
 \item Web services access layer via OGC standards for coverage access and processing
\end{itemize}

Implemented Standards

        OGC WCS 2.0, WCPS 1.0, WPS 1.0

Details

Website: www.rasdaman.org

Licence:

    clients and petascope: GNU Lesser General Public License (LGPL) version 3
    server engine: GNU General Public License (GPL) version 3

Software Version: 8.3.1

Supported Platforms: Linux, Mac, Solaris

API Interfaces: rasql (CLI), C++, Java; OGC-based WCS, WCPS, WCS-T, and WPS interfaces

Support: www.rasdaman.com
