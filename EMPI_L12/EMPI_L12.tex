% !TEX TS-program = pdflatex
% !TEX encoding = UTF-8 Unicode

% This is a simple template for a LaTeX document using the "article" class.
% See "book", "report", "letter" for other types of document.

\documentclass[12pt]{article} % use larger type; default would be 10pt

\usepackage[utf8]{inputenc} % set input encoding (not needed with XeLaTeX)
\renewcommand\familydefault{\sfdefault}
\usepackage{helvet} 


%%% Examples of Article customizations
% These packages are optional, depending whether you want the features they provide.
% See the LaTeX Companion or other references for full information.

%%% PAGE DIMENSIONS
\usepackage[a4paper,landscape,twocolumn,margin=0.5in]{geometry} % to change the page dimensions
%\geometry{a4paper} % or letterpaper (US) or a5paper or....
%\geometry{margin=0.5in} % for example, change the margins to 2 inches all round
%\geometry{twocolumn} % two columns
%\geometry{landscape} % set up the page for landscape
%   read geometry.pdf for detailed page layout information

\setlength{\columnsep}{0.5in}

\usepackage{graphicx} % support the \includegraphics command and options
\usepackage{hyperref}

% \usepackage[parfill]{parskip} % Activate to begin paragraphs with an empty line rather than an indent

%%% PACKAGES
\usepackage{booktabs} % for much better looking tables
\usepackage{array} % for better arrays (eg matrices) in maths
\usepackage{paralist} % very flexible & customisable lists (eg. enumerate/itemize, etc.)
\usepackage{verbatim} % adds environment for commenting out blocks of text & for better verbatim
\usepackage{subfig} % make it possible to include more than one captioned figure/table in a single float
% These packages are all incorporated in the memoir class to one degree or another...

%%% HEADERS & FOOTERS
\usepackage{fancyhdr} % This should be set AFTER setting up the page geometry
\pagestyle{fancy} % options: empty , plain , fancy
\renewcommand{\headrulewidth}{0pt} % customise the layout...
\lhead{}\chead{}\rhead{}
\lfoot{}\cfoot{}\rfoot{\thepage}

%%% SECTION TITLE APPEARANCE
\usepackage{sectsty}
\usepackage{titlesec}
\allsectionsfont{\sffamily\mdseries\upshape\bfseries} % (See the fntguide.pdf for font help)
% (This matches ConTeXt defaults)

%%% END Article customizations

%%% The "real" document content comes below...

\title{\large{\vspace{-5ex}\textbf{Exploration and Modelling of Planetary Interiors}}\newline\\{\textbf{\underline{Lecture 12 - Giant Planet Formation}}}\vspace{-7ex}}
\author{}
\date{} % Activate to display a given date or no date (if empty),
         % otherwise the current date is printed 

\renewcommand*\contentsname{\large\textbf{Topics covered}\vspace{-2ex}}

\bibliographystyle{apalike}
\renewcommand\refname{Background reading}

\begin{document}
\begingroup
\let\center\flushleft
\let\endcenter\endflushleft
\maketitle
\endgroup

\begingroup
\let\cleardoublepage\relax
\let\clearpage\relax
\tableofcontents
%\listoffigures
%\listoftables
\endgroup

\section{First Section}\vspace{-2ex}\titlerule[1pt]\bigskip

This assignment is worth 20\% of the marks for this module. You are
advised to start work on this assignment at the start of the module, so
that you give yourself plenty of time to complete it. The deadline for
submission will be April 25 th 2016 (the anticipated date of the revision
session for this module).\newline\linebreak
The subject of the lecture notes is a broad one, in which more than one
model has been proposed. You should first assemble a set of published
background material on the subject. You can use Google Scholar as a
resource for finding articles (insert “Giant Planet Formation” as the
topic). Alternatively, you can start with some of the references in a short
review paper by D J Stevenson which can be found at the following
website (\href{http://authors.library.caltech.edu/9922/1/STEaipcp04.pdf}{link}).\newline \linebreak
Once you have found your background material, you should consider
what topics you are going to cover in your lecture notes. Remember that
the lecture is about formation of giant planets, so you do not need to
describe giant planets and their interiors in great detail. You will also
need to illustrate your notes with suitable figures, which can be obtained
either directly from the background reading, or which can often be found
on the Web. Remember to refer to each of your figures in order in your
text. You must have a list of references at the end of your notes.\newline\linebreak
Use the same general template that we have used for our lecture notes
(landscape pages; Arial 14pt text), and please do not exceed 14 printed
pages

\subsection{A subsection}\vspace{-1ex}\titlerule[1pt]\bigskip

\cite{stevenson1982formation}



\subsection{Another subsection}\vspace{-1ex}\titlerule[1pt]\bigskip

More text.
More text.
More text.
More text.
More text.
More text.
More text.
More text.
More text.
More text.
More text.
More text.
More text.
More text.
More text.
More text.
More text.
More text.
More text.
More text.
More text.
More text.
More text.
More text.
More text.
More text.
More text.
More text.
More text.
More text.
More text.
More text.
More text.
More text.
More text.
More text.
More text.
More text.
More text.
More text.
More text.
More text.
More text.
More text.
More text.
More text.
More text.
More text.
More text.
More text.
More text.
More text.
More text.
More text.
More text.
More text.
More text.
More text.
More text.
More text.
More textMore text.
More text.
More text.
More text.
More text.
More text.
More text.
More text.
More text.
More text.
More text.
More text.
More text.
More text.
More text.
More text.
More text.
More text.
More text.
More text.
More text.
More text.
More text.
More text.
More text.
More text.
More text.
More text.
More text.
More text.
More text.
More text.
More text.
More text.
More text.
More text.
More text.
More text.
More text.
More text.
More text.
More text.
More text.
More text.
More text.
More text.
More text.
More text.
More text.
More text.
More text.
More text.
More text.
More text.
More text.
More text.
More text.
More text.
More text.
More text.
More text.
More text.
More text.
More text.
More text.
More text.
More text.
More text.
More text.
More text.
More text.
More text.
More text.
More text.
More text.
More text.
More text.
More text.
More text.
More text.
More text.
More text.
More text.
More text.
More text.
More text.
More text.
More text.
More text.
More text.
.
More text.
More text.
More text.
More text.
More text.
More text.
More text.
More text.
More text.

\newpage
\bibliography{EMPI_L12}
\end{document}
